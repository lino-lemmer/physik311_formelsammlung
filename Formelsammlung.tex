\documentclass{scrartcl}
\usepackage[ngerman]{babel}
\usepackage{paralist}
\usepackage{amsmath}
\usepackage{amssymb}
\usepackage{units}
\usepackage{kurier}
\usepackage[utf8]{inputenc}
\usepackage{commath}
\usepackage{graphicx}
\usepackage[a4paper,  left=3cm, right=2cm, top=1.53cm, bottom=2.47cm]{geometry}
\usepackage{hyperref}
\usepackage{color}
\usepackage{centernot}

\hypersetup{
	citecolor=black,
	colorlinks=true,
	linkcolor=black,
	urlcolor=blue,
}

\newcommand \e[1]{\cdot10^{#1}}
\newcommand\p{\partial}
\newcommand\half{\frac 12}
\newcommand\skp[2]{\langle#1,#2\rangle}
\newcommand\dalambert{\mathop{{}\Box}\nolimits}
\renewcommand\div[1]{\skp{\nabla}{#1}}
\newcommand\rot{\nabla\times}
\newcommand\grad[1]{\nabla#1}
\newcommand\laplace{\skp\nabla\nabla}
\renewcommand \exp[1]{\mathrm e^{#1}}
\renewcommand \i{\mathrm i}
\renewcommand{\Im}{\mathop{{}\mathrm{Im}}\nolimits}
\renewcommand{\Re}{\mathop{{}\mathrm{Re}}\nolimits}


\newcommand{\fehlt}{\textbf{\textcolor{red}{Hier fehlt Zeug...}}}

\title{physik311: Formelsammlung\\
	Wintersemester 2012/13}
\author{Lino Lemmer}

\begin{document}
\maketitle
	Dies ist eine Zusammenfassung der wichtigsten Formeln für die Klausur des Moduls \emph{Experimentalphysik III - Optik und Wellenmechanik}. Ich erhebe keinen Anspruch auf Vollständig- oder Richtigkeit der Formeln.

	\begin{center}
		\includegraphics[height=5ex]{by-sa.pdf}

		Dieses Werk bzw. Inhalt steht unter einer \href{http://creativecommons.org/licenses/by-sa/3.0/deed.de}{Creative Commons Namensnennung - Weitergabe unter gleichen Bedingungen 3.0 Unported Lizenz}.
	\end{center}


	\section{Mathematische Grundlagen}

		\subsection{Komplexe Exponentialfunktion}
			\begin{align*}
				\exp{\i x}&=\cos\del{x}+\i\sin\del{x}\\
				\abs{\exp{\i x}}&=\exp{\i x}\exp{-\i x}\\
				\sin\del x&=\frac{1}{2\i}\del{\exp{\i x}-\exp{-\i x}}\\
				\cos\del x&=\half\del{\exp{\i x}+\exp{-\i x}}
			\end{align*}

		\subsection{Fourier-Reihen-Entwicklung}
			\begin{align*}
				f\del{t}&=\frac{a_0}{2}+\sum_{n=1}^\infty a_n\cos\del{n\omega t}+b_n\sin\del{n\omega t}\\
				a_n&=\frac{2}{T}\int_c^{c+T}\dif t \, f\del{t}\cos\del{n\omega t}\\
				b_n&=\frac{2}{T}\int_c^{c+T}\dif t \, f\del{t}\sin\del{n\omega t}\\
				\text{wobei gilt:}\qquad\omega&=\frac{2\pi}{T}
			\end{align*}

		\subsection{Fourier-Transformation}
			\begin{align*}
				f\del{\omega}&=\frac{1}{\sqrt{2\pi}}\int_{-\infty}^\infty\dif t \, f\del{t}\exp{-i\omega t}\\
				f\del{t}&=\frac{1}{\sqrt{2\pi}}\int_{-\infty}^\infty\dif \omega  \, f\del{\omega}\exp{i\omega t}
			\end{align*}

	\section{Elektromagnetische Welle}
		\begin{align*}
			\vec E&=\vec E_0\exp{-\i\del{\omega t-kx}}\\
			&=\vec E_0\exp{-\i\varphi}\\
			k\del\omega&=\frac{\omega}{c}n\del\omega\\
			\lambda&=\frac{2\pi}{k}
		\end{align*}
		\textbf{Phasengeschwindigkeit:}
		\begin{align*}
			v_{\text{ph}}&=\frac{\omega}{k}
		\end{align*}
		\textbf{Gruppengeschwindigkeit:}
		\begin{align*}
			v_{\text{gr}}&=\dpd{\omega}k\\
			&=v_{\text{ph}}+k\dod{v_{\text{ph}}}k\\
			E&=cB\\
			I&\propto\abs{E\del{z,t}}^2
		\end{align*}
		\textbf{Poyntingvektor $S$:}
		\begin{align*}
			\vec S&=\vec E\times \vec H\\
			\abs{\vec S}&=I
		\end{align*}
		\textbf{Fresnel'sche Formeln:} \\
		\hfill \\
		Der Index $p$ bezeichnet parallel-, $s$ senkrecht-polarisiertes Licht, die Indices $i$, $r$ und $t$ bezeichnen die einfallende, transmittierte und reflektierte Amplitude von $E$:
		\begin{align*}
			\del{\frac{E_{0r}}{E_{0i}}}_s &=\frac{n_1\cos\theta_i-n_2\cos\theta_t}{n_1\cos\theta_i+n_2\cos\theta_t}\\
			\del{\frac{E_{0t}}{E_{0i}}}_s &=\frac{2n_1\cos\theta_i}{n_1\cos\theta_i+n_2\cos\theta_t}\\
			\del{\frac{E_{0r}}{E_{0i}}}_p &=\frac{n_2\cos\theta_i-n_1\cos\theta_t}{n_1\cos\theta_i+n_2\cos\theta_t}\\
			\del{\frac{E_{0t}}{E_{0i}}}_p &=\frac{2n_1\cos\theta_i}{n_1\cos\theta_i+n_2\cos\theta_t}
			\end{align*}
		\textbf{Reflexivität bei senkrechtem Einfall:}
		\begin{align*}
			\frac{E_r}{E_i}&=\frac{n_1-n_2}{n_1+n_2}\\
			R&=\del{\frac{n_1-n_2}{n_1+n_2}}^2
		\end{align*}

	\section{Strahlenoptik}
		\textbf{Lichtdruck}
			\begin{align*}
				P_S&=\frac{I}{c}=\frac{W}{Ac}
			\end{align*}

		\subsection{Brechung}
			\textbf{Fermat'sches Prinzip:} \\
			Licht nimmt im Medium den kürzesten optischen Weg.
			\begin{align*}
			\intertext{\textbf{Brechungsindex $n$}}
				n&=\frac{c}{v_{ph}}
			\intertext{\textbf{Brechungsgesetz von Snellius}}
				\frac{n_1}{n_2}&=\frac{\sin\varphi_2}	{\sin\varphi_1}
			\intertext{Für die Totalreflexion folgt hieraus}
				\sin\varphi_T&=\frac{n_1}{n_2}\qquad\text{, mit }n_1<n_2\\
			\intertext{Der Winkel unter dem nur der senkrecht zur Einfallsebene polarisierte Teil der Welle reflektiert wird (Brewster-Winkel):}
				\varphi_B&=\arctan\del{\frac{n_2}{n_1}}\quad,\quad n_1<n_2\quad,\quad \varphi_1+\varphi_2=\varphi_B+\varphi_2=\frac{\pi}{2}
			\end{align*}

		\subsection{Hohlspiegel}
			Abbildungsgleichung im Hohlspiegel
			\begin{align*}
				\frac{1}{g}+\frac{1}{b}&=\frac{1}{f}
			\intertext{mit}
				f&=\frac{r}{2}
			\intertext{Hier ist bei konkavem Spiegel $r$ positiv.\newline Reelle Bilder gibt es für $r>0$ bei folgenden Konstellationen}
				\frac{B}{G}&<1\qquad\text{, für }g>2f\\
				\frac{B}{G}&=1\qquad\text{, für }g=2f\\
				\frac{B}{G}&>1\qquad\text{, für }f<g<2f
			\intertext{Virtuelle Bilder gibt es bei folgender Konstellation}
				\frac{B}{G}&<0\qquad\text{, für }g<f
			\end{align*}

		\subsection{Kugelflächen}
			Abbildungsgleichung für Brechung an Kugelflächen. Hier ist $r>0$ für konvexe Kugelflächen
			\begin{align*}
				\frac{n_1}{g}+\frac{n_2}{b}&=\frac{n_2-n_1}{r}
			\intertext{Mit der hinteren Brennweite $f$}
				f&=\frac{n_2}{n_2-n_1}r
			\intertext{und der vorderen Brennweite $F$}
				F&=\frac{n_1}{n_2-n_1}r
			\intertext{erhält man die Abbildungsgleichung}
				\frac{F}{g}+\frac{f}{b}&=1
			\intertext{Bei mehreren Brechungen addieren sich die Brechkräfte, oder auch die reziproken Brennweiten}
				\frac{1}{f}&=\frac{1}{f_1}+\frac{1}{f_2}
			\intertext{Im Fall einer dünnen Linse erhält man (Linsenschleifergleichung)}
				\frac{1}{f}&=\del{n-1}\del{\frac{1}{r_1}-\frac{1}{r_2}}
			\end{align*}
			Folgende Linsenfehler gibt es:
			\begin{itemize}
			\item
				Koma: Schief einfallende Strahlen werden, da sie unterschiedliche Bereiche der Linse durchlaufen nicht exakt in einem Punkt abgebildet.
			\item
				Astigmatismus: Schief einfallende Strahlen werden je nach Ebene unterschiedlich gebündelt, daher ist die Form der Querschnittsfläche des Strahlenbündels nicht konstant.
			\item
				sphärische Aberration: Strahlen, die einen unterschiedlichen Abstand zur optischen Achse haben, werden unterschiedlich stark gebrochen. Folge: Kaustik.
			\item
				chromatische Aberration: Aufgrund des unterschiedlichen Brechungsindex bei unterschiedlichen Wellenlängen, werden verschiedene Farben verschieden stark gebrochen.
			\end{itemize}

	\section{Matrizenoptik}
 		Allgemein: Darstellung des Strahlengangs durch Vektor $\begin{pmatrix}r\\\varphi\end{pmatrix}$, wobei $r$ den Abstand und $\varphi$ den Winkel zur optischen Achse bezeichnet.\\
 	Die Änderung von $r$ und $\varphi$ durch ein optisches Instrument kann durch eine $2\times2$ Transformationsmatrix (TFM) dargestellt werden. Die TFM der Kombination mehrerer optischen Instrumente ergibt sich dann aus der Multiplikation der einzelnen TFM.

 		\subsection{Transformationsmatrizen}
 			Lineare Bewegung:
 			\begin{align*}
 				\begin{pmatrix}
 					1 & d\\
 					0 & 1
 				\end{pmatrix}&
 			\intertext{\textbf{Reflexion am sphärischen Hohlspiegel}}
 				\begin{pmatrix}
 					1 & 0\\
 					-\frac{2}{r} & 1
 				\end{pmatrix}&=
 				\begin{pmatrix}
 					1 & 0\\
 					-\frac{1}{f} & 1
 				\end{pmatrix}
 			\intertext{\textbf{Brechung an ebener Fläche}}
 				\begin{pmatrix}
 					1 & 0\\
 					0 & \frac{n_2}{n_1}
 				\end{pmatrix}
 			\intertext{\textbf{Brechung an Kugelfläche}}
 				\begin{pmatrix}
					1 & 0\\
					\del{\frac{n_1}{n_2}-1}\frac{1}{r} & \frac{n_1}{n_2}
 				\end{pmatrix}
 			\intertext{\textbf{Dünne Linse in Luft}}
 				\begin{pmatrix}
 					1 & 0\\
					-\del{n-1}\del{\frac{1}{r_1}-\frac{1}{r_2}} & 1
 				\end{pmatrix}&=
 				\begin{pmatrix}
 					1 & 0\\
 					-\frac{1}{f} & 1
 				\end{pmatrix}
 				\intertext{Abstand der Hauptebenen $H_b$ und $H_g$ zur Mitte des optischen Systems mit der Matrix $M$:}
 				M&=\begin{pmatrix}
 				A & B\\
 				C & D
 				\end{pmatrix}\\
 				h_g&=\del{1-D}f\\
 				h_b&=\del{1-A}f
 			\end{align*}

 	\section{Interferenz und Beugung}

 		\subsection{Doppelspalt}
 			Mit
 			\begin{align*}
 				s&=\text{Abstand der Spalte}\\
 				\lambda&=\text{Wellenlänge des Lichts}\\
 				\alpha_{\text{max/min}}&=\text{Winkelabstand zum Maximum/Minimum}\\
 				d_{\text{max/min}}&=\text{Abstand zum Maximum/Minimum}\\
 				D&=\text{Abstand zum Schirm}\\
 				n&\in\mathbb{Z}
 			\intertext{ergibt sich}
 				\alpha_{\text{max}}&=\arcsin\del{\frac{n\lambda}{s}}\approx\frac{n\lambda}{s}\\
 				d_{\text{max}}&\approx D\frac{n\lambda}{s}\\
 				\alpha_{\text{min}}&= \arcsin \del{\frac{\del{n+\half}\lambda}{s}} \approx\frac{\del{n+\half}\lambda}{s}\\
 				d_{\text{min}}&\approx D\frac{\del{n+\half}\lambda}{s}
 			\end{align*}

 		\subsection{Gitter}
 			Bedingung für Maximum: Siehe Doppelspalt.\newline
 			Zusätzlich gilt mit
 			\begin{align*}
 				N&=\text{Anzahl der Gitter}\\
 				n&=\text{Anzahl der Nebenmaxima}\\\\
 				n&=N-2
 			\end{align*}

 		\subsection{Einzelspalt}
 			Mit
 			\begin{align*}
 				b&=\text{Breite des Spaltes}\\
 				\lambda&=\text{Wellenlänge des Lichts}\\
 				\alpha_{\text{max/min}}&=\text{Winkelabstand zum Maximum/Minimum}\\
 				d_{\text{max/min}}&=\text{Abstand zum Maximum/Minimum}\\
 				D&=\text{Abstand zum Schirm}\\
 				n&\in\mathbb{Z}
 			\intertext{ergibt sich}
 				\alpha_{\text{min}}&=\arcsin\del{\frac{n\lambda}{b}}\approx\frac{n\lambda}{b}\\
 				d_{\text{min}}&\approx D\frac{n\lambda}{b}\\
 				\alpha_{\text{max}}&= \arcsin \del{\frac{\del{n+\half}\lambda}{b}} \approx\frac{\del{n+\half}\lambda}{b}\\
 				d_{\text{max}}&\approx D\frac{\del{n+\half}\lambda}{b}
 			\end{align*}

 		\subsection{Fabry-Perot-Resonator}
 			\textbf{Freier Spektralbereich}
 			\begin{align*}
 				\Delta\nu&=\frac{c}{2d}
 			\intertext{\textbf{Finesse und Reflexivität}}
 				\mathcal{F}&=\frac{\pi\sqrt{R}}{1-R}
 			\end{align*}

 		\subsection{Michelson-Interferometer}
 			Die Intensität in Abhängigkeit vom Ort des bewegten Spiegels. Kein Gangunterschied $\delta$ bei $t=0$.
 			\begin{align*}
 				I\del t&=\frac{1}{16}I_0\abs{1+\exp{-\i k\delta\del t}}^2\\
 				&=\frac{1}{16}\del{1+\exp{-\i k\delta}}\del{1+\exp{\i k\delta\del t}}\\
 				\delta\del t&=2vt
 			\end{align*}

		\subsection{Auflösungsvermögen optische Instrumente}
			Mit
			\begin{align*}
				\alpha_{\text{min}}&=\text{minimaler auflösbarer Winkel}\\
				d&=\text{Durchmesser der Öffnung}\\
				\lambda&=\text{Wellenlänge des Lichts}
			\intertext{ergibt sich das Rayleigh-Kriterium zu}
				\alpha_{\text{min}}&=1.22\frac{\lambda}{d}
			\intertext{Die numerische Apertur ist}
				A_N&=1.22\frac{\lambda}{2d}
			\intertext{für eine Linse gilt die Näherung}
				&=\frac{d}{2f}
			\end{align*}

	\section{Polarisation}
		\textbf{Elektrische Welle:}
		\begin{align*}
			\vec E&= E_{0x}\vec{e}_x\exp{\i\phi_x} + E_{0y}\vec{e}_y\exp{\i\phi_y}=
			\begin{pmatrix}
				E_{0x}\exp{\i\phi_x}	\\
				E_{0y}\exp{\i\phi_y}
			\end{pmatrix}
		\intertext{Nun kann man vier Fälle unterscheiden:}
		\phi_x-\phi_y&=2\pi &\implies\text{linear polarisierte Welle}\\
		\phi_x-\phi_y&=n2\pi+\frac{\pi}{2}\quad, E_{0x}=E_{0y}\quad &\implies\text{linkszirkular polarisierte Welle}\\
		\phi_x-\phi_y&=n2\pi-\frac{\pi}{2}\quad, E_{0x}=E_{0y}&\implies\text{linkszirkular polarisierte Welle}\\
		\phi_x-\phi_y&\neq 2\pi-\frac{\pi}{2}\quad \lor E_{0x}\neq E_{0y}&\implies\text{elliptisch polarisierte Welle}\\
		\intertext{Malus'sches Gesetz (Intensität einer linear polarisierten Welle nach Durchgang durch einen Linearpolarisator der im Winkel $\alpha$ zur Polarisation der Welle steht:}
			I&=I_0\cos^2\alpha
		\end{align*}

		\subsection{Doppelbrechung}
			\textbf{Optisch einachsiger Kristall:}
			\begin{align*}
				n_o\neq n_{ao}\\
				n_o<n_{ao}\quad\implies\quad\text{optisch positiv}\\
				n_o>n_{ao}\quad\implies\quad\text{optisch negativ}
			\end{align*}
			\fehlt

		\subsection{Jones Formalismus}
			\subsubsection{Jones-Vektor}
				\begin{align*}
					\vec{J}&=\frac{1}{\abs{\vec{E}}}
					\begin{pmatrix}
						E_{0,x}\cdot\exp{\i\phi_x}\\
						E_{0,y}\cdot\exp{\i\phi_y}
					\end{pmatrix}
				\intertext{Beispiele:\newline
				linear in $x$-Richtung polarisiertes Licht:}
					\begin{pmatrix}
						1\\0
					\end{pmatrix}
				\intertext{linear in $y$-Richtung polarisiertes Licht:}
					\begin{pmatrix}
						0\\1
					\end{pmatrix}
				\intertext{linear polarisiertes Licht mit $E_{0,y}=aE_{0,x}$}
					\frac{1}{\sqrt{a+1}}
					\begin{pmatrix}
						1\\a
					\end{pmatrix}
				\intertext{linkszirkular polarisiertes Licht}
					\frac{1}{\sqrt{2}}
					\begin{pmatrix}
						1\\\i
					\end{pmatrix}
				\intertext{rechtszirkular polarisiertes Licht}
					\frac{1}{\sqrt{2}}
					\begin{pmatrix}
						1\\-\i
					\end{pmatrix}
				\end{align*}

			\subsubsection{Jones-Matrix}
				Linearpolarisator in $x$-Richtung
				\begin{align*}
					\begin{pmatrix}
						1 & 0\\
						0 & 0
					\end{pmatrix}&
				\intertext{Linearpolarisator in $y$-Richtung}
					\begin{pmatrix}
						0 & 0\\
						0 & 1
					\end{pmatrix}
				\intertext{Linearpolarisator allgemein}
					\begin{pmatrix}
						\cos^2\phi & \cos\phi\sin\phi\\
						\sin\phi\cos\phi & \sin^2\phi
					\end{pmatrix}&
				\intertext{Polarisator für linkszirkulares Licht}
					\half \begin{pmatrix}
						1 & -\i\\
						\i & 1
					\end{pmatrix}&
				\intertext{Polarisator für rechtszirkulares Licht}
					\half \begin{pmatrix}
						1 & \i\\
						-\i & 1
					\end{pmatrix}&
				\intertext{$\frac{\lambda}{2}$-Plättchen mit schneller Achse in x-Richtung}
					\begin{pmatrix}
						-\i & 0\\
						0 & \i
					\end{pmatrix}&=
					\begin{pmatrix}
						\exp{-\frac{\pi}{2}} & 0\\
						0 & \exp{\frac{\pi}{2}}
					\end{pmatrix}
				\intertext{$\frac{\lambda}{4}$-Plättchen mit schneller Achse in x-Richtung}
					\frac{1}{\sqrt{2}}\begin{pmatrix}
						1-\i & 0\\
						0 & 1+\i
					\end{pmatrix}&=
					\begin{pmatrix}
						\exp{-\i\frac{\pi}{4}} & 0\\
						0 & \exp{\i\frac{\pi}{4}}
					\end{pmatrix}
				\intertext{Sind Bauteile $M$ um $\theta$ gedreht, erhält man die Jones-Matrix durch}
					M\del\theta=R\del\theta M R\del{-\theta}&\qquad ,\text{ mit }\quad R\del\theta=
					\begin{pmatrix}
						\cos\theta & -\sin\theta\\
						\sin\theta & \cos\theta
					\end{pmatrix}
				\end{align*}

	\section{Quantenoptik}
		\textbf{Energie von Lichtquanten}
		\begin{align*}
			E&=hf=\frac{hc}{\lambda}
		\intertext{\textbf{Unschärferelation}}
			\Delta p\cdot\Delta x&\geq h
		\intertext{\textbf{für statistische Schwankungen}}
			\sigma_p\cdot\sigma_x&\geq \frac{\hbar}{2}
		\intertext{\textbf{De-Broglie-Wellenlänge}}
			\lambda_{\text{dB}}&=\frac{h}{p}
		\intertext{\textbf{Stefan-Boltzmann-Gesetz}}
			P&= 4\pi r^2\sigma T^4
		\intertext{\textbf{Wiensches Verschiebungsgesetz}}
			\lambda_{\text{max}}&=\frac{\unit[2.8978\e{-3}]{m\,K}}{T}
		\end{align*}
 \end{document}
